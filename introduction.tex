%%%%%%%%%%%%%%%%%%%%%%%%%%%%%%%%%%%%%%%%%%%%%%%%%%%%%%%%%%%%%%%%%%%%%%%%
% Introduction
%%%%%%%%%%%%%%%%%%%%%%%%%%%%%%%%%%%%%%%%%%%%%%%%%%%%%%%%%%%%%%%%%%%%%%%%
\section{Introduction}
\label{sec:intro}

%What is the problem?
% see here for description why data quality is important.
% http://books.google.de/books?hl=en&lr=&id=SULMBFgtwQoC&oi=fnd&pg=PA1&ots=uaNGyVfL7V&sig=RER31QotCCRJeaQq8OafAtpgK1k#v=onepage&q&f=false
Having access to high quality data is of great importance in data analysis. However, data in the real world is often considered \textit{dirty}: it contains inaccurate, incomplete, inconsistent, duplicated, or stale values~\cite{chu2004blissful}. A number of distinct data quality issues are known in the field of data quality management such as \textit{data consistency}, \textit{currency}, \textit{accuracy}, \textit{deduplication} and \textit{information completeness}~\cite{fan2012foundations}. As previous work has observed, such data quality issues are detrimental to data analysis~\cite{national2013Frontiers,Fan:2008:CFD:1366102.1366103} and cause huge costs to busi\-nesses \cite{waynew.eckerson2002}. Therefore, improving data quality with respect to business and integrity constraints is a crucial component of data management. 
A common approach to increase data quality is to formulate a set of \textit{data cleaning rules} that detect semantic errors by utilizing data dependencies~\cite{fan2012foundations, Arasu:2009:LDC:1546683.1547340, Dallachiesa:2013:NCD:2463676.2465327, llunaticVDLB2013b}. However, previous research identified a number of requirements and accompanying challenges, which are associated with creating such rule sets (c.f., Section~\ref{sec:frontiers}): 

%\todo[inline]{Desiderata: Challenge}  
\textit{Interleaved rules}. First, while each such rule usually addresses one data quality issue individually, the individual rules as a whole typically \textit{interact}~\cite{fan2012foundations, Fan:2014:IRM:2628135.2567657}. For instance, a rule that deletes duplicates might perform better after missing data has already been imputed, while, on the other hand, a rule that imputes missing data might perform better if duplicates have already been removed. Therefore, we argue to model data quality rules such as deduplication and missing value imputation \textit{jointly}, rather than as separate processes.
Second, rules in such a rule-set may need to be modeled \textit{"soft"} and \textit{"hard"} in order to balance constraints of different importance \cite{Yakout:2013:DSU:2463676.2463706}, especially within a set of interacting rules. 

\textit{Automation}. Different execution orders of interleaved rules produce different results~\cite{Dallachiesa:2013:NCD:2463676.2465327}. Imposing the difficult problem of manually specifying the execution order on the user conflicts with the automation principle of data curation systems~\cite{Stonebraker_datacuration}.

\textit{Usability and domain knowledge integration}. Various languages and statistical approaches for data curation exist \cite{Dallachiesa:2013:NCD:2463676.2465327, chu2013holistic, llunaticVDLB2013b}. However, there is a need for expressiveness and customization of the rules in order to integrate arbitrary constraints into data cleaning without having to specify complex user-defined functions. 

In this paper, we present an approach to data cleaning based on statistical relational learning (SRL)~\cite{getoor2007introduction} and probabilistic inference~(c.f.,~Section~\ref{sec:method}). SRL is a branch of machine learning that models joint distributions over relational data. Generally, data quality rules represent relationships between attributes in the database schema (c.f., Section~\ref{sec:expl}). These rules are mainly based on integrity constraints such as functional dependencies~\cite{AbiteboulHV95, fan2012foundations} on top of a database schema. We show how to translate such functional dependencies, expressed as first-order logic formulas, into probabilistic-logical languages, which allows us to reason over inconsistencies, duplicates or missing values in a probabilistic way~(c.f.,~Section~\ref{sec:ml}). During automatic data cleaning, the optimal order of rules execution is hardly achievable \cite{Dallachiesa:2013:NCD:2463676.2465327}. Therefore, we propose to use joint inference for the simultaneous rules execution.

The contributions of this paper are the following:

\begin{itemize}
  \item We discuss how to model data cleaning rules based on integrity constraints within the probabilistic framework of\\ Markov~logic~(Section~\ref{sec:ml}).
  \item We describe how data repair leverages joint inference on probabilistic graphical models and allows us to treat data curation rules holistically, which obliterates the need to specify execution order (Section~\ref{subsec:jointinference}).\\
 \item We present an extensive empirical study of holistic modeling and error prediction in data cleaning using Markov logic. Modelling data quality rules in a probabilistic way enables statistical inference of data quality errors. Our experimental evaluation reveals that the simultaneous treatment of data cleaning rules leads to higher accuracy in data curation. We show that joint inference for data correction prediction outperforms sequential execution of data cleaning rules on a dataset comprised of hundred thousands tuples with $10\%$ of noisy values, and results in an improved $F_1$-score of $0.95$ (Section~\ref{sec:evaluation}).
\end{itemize}

