%%%%%%%%%%%%%%%%%%%%%%%%%%%%%%%%%%%%%%%%%%%%%%%%%%%%%%%%%%%%%%%%%%%%%%%%
% Data Cleaning Rules
%%%%%%%%%%%%%%%%%%%%%%%%%%%%%%%%%%%%%%%%%%%%%%%%%%%%%%%%%%%%%%%%%%%%%%%%
\section{Fundamentals}
\label{sec:expl}

\note{This section is NOT a contribution and reads way too difficult. We need to shorten it and make it easier to read, so that reviewers easily reach Section 3}

\subsection{Data Quality}

We consider a database instance $\mathcal{D}$ with a relational schema $\mathcal{S}$. Furthermore, we consider relation $\mathcal{R} \in \mathcal{S}$ that is defined for the set of attributes $attr(\mathcal{R})$. Moreover, $dom(U_i)$ denotes the domain of an $i$-th attribute $U_i \in attr(\mathcal{R})$. For the data cleaning, we consider a set of data quality rules, which are data-agnostic and rely on data integrity constraints~\cite{AbiteboulHV95}. In the following we study data cleaning rules based on \emph{Integrity Constraints}.

A \textbf{functional dependency} (FD) on a relational schema $S$ is an expression of the form $\phi: X \rightarrow Y$ where $X \subseteq attr(\mathcal{R}) $ and $Y \subseteq attr(\mathcal{R}) $ are subsets of $\mathcal{R}'s$ attributes $attr(\mathcal{R})$. This FD holds if every pair of tuples of $\mathcal{R}$ that agree in each of attributes $X$, also agree in the $Y$ attributes. Functional dependencies are used by data quality management systems to find and correct dirty data, because these dependencies specify the semantics of the data in a declarative way. The main disadvantage of functional dependencies is that they operate solely on the schema level and are often not able to detect conflicts in real-life data \note{why?}.Therefore, we consider an  extension of traditional functional dependencies:  A \textbf{conditional functional dependency} (CFD) defined on the relational schema $\mathcal{S}$ is a pair $\psi(X \rightarrow Y , T_p)$,  where $X \rightarrow Y$ is a standard FD and $X \cup Y \in attr(\mathcal{R})$ and $T_p$ is a tuple pattern, which is a set of constraints holding on the particular subset $X \cup Y$ of tuples. For each $U \in X \cup Y$, the value of the attribute $U$ for $T_p$, $T_p[U]$ is either a variable value $`\_`$, or a constant $u \in dom(U)$. \anno{DO WE REALLY NEED THIS INFO HERE? We assume that CFDs are provided in the normal form. This means if $\psi(X \rightarrow Y_1,Y_2,\dots , T_p)$, then $\psi$ will be decomposed into several CFDs where $RHS(\psi)$ (right hand side of $\psi$) becomes a single attribute: $\psi_1(X \rightarrow Y_1 , T_p)$, $\psi_2(X \rightarrow Y_2 , T_p) \dots$. Furthermore, we distinguish between \textit{constant} and \textit{variable} CFDs. The former contains tuple pattern $T_p$ where $T_p[Y]\neq `\_`$. Otherwise, $\psi$ is called \textit{variable}.} Data deduplication techniques use FDs to define the matching dependencies. The difference here is that matching dependencies use the notion of the similarity predicate instead of equality predicate. A \textbf{Matching Dependency} (MD) for schemas $\mathcal{S}_1$ and $\mathcal{S}_2$ is syntacticly defined as:
  $\mu: \mathcal{S}_1[X_1]\approx \mathcal{S}_2[X_2]\rightarrow \mathcal{S}_1[Y_1]\rightleftharpoons \mathcal{S}_2[Y_2]$ 
  where $X_1 \cup Y_1$ and $X_2 \cup Y_2$ are pairwise compatible sets of attributes in $attr(\mathcal{R}_1), \mathcal{R}_1\in \mathcal{S}_1$ and $attr(\mathcal{R}_2), \mathcal{R}_2\in \mathcal{S}_2$, respectively; $\approx$ indicates
  similar attributes and $\rightleftharpoons$ is called the \textit{matching operator}. In other words the matching operator $\rightleftharpoons$ means that for each $\mathcal{S}_1$ tuple $t_1$ and each $\mathcal{S}_2$ tuple $t_2$: $t_1[Y_1]$ and $t_2[Y_2]$ refers to the same real-world entity. Having dynamic semantics, that is pointing \textit{how} to repair data, MD forces to update $t_1[Y_1]$ and $t_2[Y_2]$ such that they have the same values. The operator $\rightleftharpoons$ only requires that the $t_1[Y_1]$ and $t_2[Y_2]$ are identified.

\subsection{Markov Logic}

\anno{This subsection needs to be connected to the rest of the paper, at the moment, its totally not understandable!!!}

\textbf{Markov logic}~\cite{domingos2009markov} is a knowledge representation system that combines first-order logic as a declarative language and probabilistic graphical models (undirected Markov Networks) on top of which probabilistic inference is performed. \anno{CONFUSING, REWRITE OR REMOVE: Markov Networks can be viewed in terms of \emph{cliques} which are subgraphs where all pairs of nodes are connected.} Edges between nodes are parametrized by so-called potentials, to encode the strength of a conditional dependency. \note{What do edges and nodes represent?} \anno{CONFUSING, REWRITE OR REMOVE: The joint probability of a Markov Network is attained by computing the normalized product over clique potentials.} 
%\todo[inline]{markov logic into from the Rockit paper section: "Markov Logic"; \\ also explain $F_i$ below - adopt from ELEMENTARY paper page 7: semantics}
Semantically, a Markov Logic Network (MLN) is a log-linear model, which defines the probability 
distribution over possible worlds, in our case all possible states in the database.% Such models have many names, including maximum-entropy models, exponential models, and Gibbs models; Markov random fields are structured log-linear models, conditional random fields (Lafferty et al., 2001) are Markov random fields with a specific training criterion
We consider a set of formulae $\bar{F} = F_1, F_2 \dots F_N$ with corresponding weights $w_1 \dots w_N$ as an input. \note{Why? What do these formulas represent?} These weighted first-order logic formulae define a probability distribution as follows: $P \left( X = x \right) = \frac{1}{Z} \exp\left( \sum_{F_i} w_i n_i \left( x \right) \right)$. Here, $n_i(x)$ denotes the number of true groundings of formula $F_i$ in $x$. The term $w_i$ represents the corresponding formula weight and $Z$ is a normalization constant. 
Given a formula $F_i$ with variables $(x_1 \dots x_n)$, we create new formula $g$ called \textit{ground formula} \note{Why?} as follows: for each $x_k$ and its domain $dom(U_k)$, each variable $x_k$ is being substituted with a value from its domain $dom(U_k)$.
Let $w$ be a function that maps each ground formula $g$ to its assigned weight. If $g$ is false (true) and $w(g)>0$ (respectively $w(g)<0$), then the ground formula is violated. Hence, the probability of the particular world is higher the less ground formulae is violated and the lower is the cost of the world $W$: \note{Don't talk about 'worlds', connect this to our task}. $cost_W = \sum_{g \in V(W)} w(g)$. Here $V(W)$ denotes the set of violated ground formulae.

\anno{CONFUSING, REMOVE: Please note that formula weights are not probabilities itself but the certainty of this formula. This also means that weights 
can be considered as empirical frequencies where each formula should hold. Weights can be specified in two ways: first, heuristically (e.g using the domain expertise); second, by performing weight learning on data, e.g. by running weight learning algorithms on MLN \cite{lowd2007efficient}.}

\note{CLARIFY: In this paper we focus on an inference task as a prediction for the data cleaning operations.}

