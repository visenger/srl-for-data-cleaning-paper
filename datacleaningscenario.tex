\section{Frontiers in Data Cleaning}
\label{sec:frontiers}

Next, we showcase limitations of rule-based data cleaning. Therefore, we introduce the following data cleaning scenario:
\label{sec:example}
% CUSTOMER
\begin{table}[h]\footnotesize
\scriptsize
\resizebox{\columnwidth}{!}{%
\begin{tabular}{lllllll} \toprule 
\textbf{id} &  \textbf{firstname} & \textbf{lastname} & \textbf{street} & \textbf{city} & \textbf{zipcode} & \textbf{phone} \\ \midrule
$c_1$ & Ron & Howard & 1 Sun Dr. & Los Angeles & 90001 & 12345 \\
$c_2$ & Max & Miller & 12 Hay St. & Napa & 94558 & 11234 \\ \bottomrule
\end{tabular}}
\vspace{-1em}
\caption{\textsc{customer} table (with errors)}
\label{tab:cust}
\end{table}

% TRANSACTION
\begin{table}[h]\footnotesize
\scriptsize
\resizebox{\columnwidth}{!}{%
\begin{tabular}{llllllll} \toprule 
\textbf{id} & \textbf{item} &  \textbf{firstname} & \textbf{lastname} & \textbf{street} & \textbf{city} & \textbf{zipcode} & \textbf{phone} \\ \midrule
$t_1$ & iPhone6 & R. & Howard & 1 Sun Dr. & L.A. & null & null \\
$t_2$ & Galaxy5 & null & Miller & 12 Hay St. & null & 94558 & 11234 \\
$t_3$ & Nexus5 & Howard & Ron & null & null & 90001 & 12345 \\ \bottomrule
\end{tabular}}
\vspace{-1em}
\caption{\textsc{transaction} table (with errors)}
\label{tab:trans}
\end{table}


Consider the following relational tables: the \textsc{customer} table (c.f.,~Table~\ref{tab:cust}) 
records address and contact details of each customer. The \textsc{transaction} table (c.f.,~Table~\ref{tab:trans}) 
records each purchased item, together with the personal details entered by the customer during the purchase. 
The example data has several quality issues: (1) Missing values in the \textsc{transaction} table, indicated by \emph{null} values; (2) false values, e.g.,  the customer ``Ron Howard'' is involved in transaction $t_3$ , but in the transaction table his name is falsely recorded as ``Howard Ron''; (3) duplication issues, e.g., the city of ``Los Angeles'' is sometimes entered into the table as ``L.A.'' and the first name ``Ron'' is once abbreviated as ``R.''. 

%Note: I would not call (3) as duplication, since it is across different tables. 

In order to correct these issues, we need to identify the corresponding entries for each transaction in the \textsc{customer} table and use these to automatically clean the transaction data. Although enough information exists to make such links possible through manual linking, defining automatic cleaning rules to accomplish this task is not trivial. For instance, a hard rule that states that $\textsl{city}$ field values ``L.A.'' and ``Los Angeles'' are always the same, intuitively makes sense. Yet, this assumption does not hold for the $\textsl{firstname}$ values ``R.'' and ``Ron''. Rather, we would need a \emph{soft rule} that indicates that both strings \emph{possibly} refer to the same entity, but that we may require additional evidence. Next, consider the \emph{null} value in $t_1.$\textsl{zipcode}. We would like to fill this missing value by linking the transaction to customer $c_1$, but only the fields \textsl{lastname} and \textsl{street} fully match, while \textsl{city} (and possibly \textsl{firstname}) may match using the de-duplication rules mentioned above. Even though this provides significant evidence that the transaction should be linked to customer $c_1$, there is still a chance that two different people share the same last name and street address in a large city. Inclusion of the \textsl{phone} field into the rule would make us reasonably certain that they should be linked, but as the example shows, the value for this field is missing. 

This example illustrates the limitations of data cleaning via hard rules: Because the example table is heavily corrupted, it is very hard to define data cleaning rules, which do not introduce errors. Instead, we would prefer to define a number of \emph{soft rules} that aggregate evidence. Examples include first name abbreviation rule (``R.'' vs. ``Ron''), a first and last name switch rule (indicating that there is a small chance that ``Ron Howard'' and ``Howard Ron'' are the same person), and rules that indicate that the more fields two tuples in the \textsc{customer} and \textsc{transaction} tables share, the more likely it is that they refer to the same person. 

The functional dependency: $\textsl{fd}: \textsc{transaction}([\textsl{city}, \textsl{phone}] \rightarrow [\textsl{street}, \textsl{zipcode}])$ declares that, in the \textsc{transaction} table, the two fields \textsl{city} and \textsl{phone} together 
uniquely determine the two fields \textsl{street} and \textsl{zipcode}. Although two instances of the customer ``Ron Howard'' in ~Table~\ref{tab:trans} are recorded, one is missing the \textsl{phone} value. In this case, the rule only applies when combined with additional data cleaning rules. The rule $\textsl{cfd}: \textsc{transaction}([\textsl{zipcode}] \rightarrow [\textsl{city}], T_1 =(90001~||~\text{Los~Angeles}))$ states that every tuple in which the value for \textsl{zipcode} equals $90001$ must have its \textsl{city} attribute set to ``Los~Angeles". In our example in Table~\ref{tab:trans}, this rule corrects the \emph{null} value in transaction $t_3$. 

For the \textsl{firstname} field in transaction $t_1$ in Table~\ref{tab:trans}, we could define a rule indicating that the values ``R.'' and ``Ron'' refer to the same name, but cannot always be certain whether this is the case. To include additional evidence, we create the following MD to link transaction $t_1$ to customer $c_1$:\\ 
%$ \textsl{md}: \textsc{transaction}[\textsl{lastname}, \textsl{city}, \textsl{street}] = \textsc{customer}[\textsl{lastname}, \textsl{city}, \textsl{street}] \land \textsc{t}[\textsl{firstname}] \approx \textsc{customer}[\textsl{firstname}]  \rightarrow \textsc{transaction}[\textsl{firstname}] \rightleftharpoons \textsc{customer}[\textsl{firstname}]$
\vspace*{-0.5cm}
\begin{table}[h]\footnotesize
\scriptsize
\centering
\resizebox{\columnwidth}{!}{%
%\begin{tabular}{l}
\begin{tabular}[c]{@{}c@{}}
$ \textsl{md}: \textsc{transaction}[\textsl{lastname}, \textsl{city}, \textsl{street}]
= \textsc{customer}[\textsl{lastname}, \textsl{city}, \textsl{street}] $ \\
$\land \textsc{transaction}[\textsl{firstname}] \approx \textsc{customer}[\textsl{firstname}] $ \\
$ \rightarrow \textsc{transaction}[\textsl{firstname}] \rightleftharpoons \textsc{customer}[\textsl{firstname}]$

%\end{tabular}
\end{tabular}}
\end{table}
\vspace*{-0.5cm}

This MD states that if any two tuples from \textsc{transaction} and \textsc{customer} match on \textsl{lastname}, \textsl{city}, \textsl{street}, and their \textsl{firstname} values are similar, then \textsc{transaction}[\textsl{firstname}] and \textsc{customer} [\textsl{firstname}] should have the same values. This rule shows how the notion of similarity may be included in rules, but only within first-order logic, i.e., the similarity condition is either \emph{true} or \emph{false}. In reality, we are able to determine a more fine graded similarity, i.e., some types of similarity that we deem to be more probable (``L.A.'' and ``Los Angeles''), and others that are less probable (``R.'' and ``Ron''). Moreover, we may have different levels of confidence regarding functional or matching dependencies that we define. The above rule for the example is very likely, but not necessarily always true. However, hard first-order logic does not permit us to formalize this intuition, as it does not allow for a probabilistic interpretation of the rules.