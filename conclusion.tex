
%%%%%%%%%%%%%%%%%%%%%%%%%%%%%%%%%%%%%%%%%%%%%%%%%%%%%%%%%%%%%%%%%%%%%%%%
% Conclusion
%%%%%%%%%%%%%%%%%%%%%%%%%%%%%%%%%%%%%%%%%%%%%%%%%%%%%%%%%%%%%%%%%%%%%%%%
\section{Conclusion and Future Work}
\label{sec:conclusion}
We presented a declarative data cleaning approach based on statistical relational learning and probabilistic inference. We demonstrated how functional dependencies, expressed as first-order logic formulas, are translated into probabilistic logical languages, allowing us to reason over inconsistencies or duplicates in a probabilistic way. Our approach allows the usage of probabilistic joint inference over interleaved data cleaning rules to improve data quality. By using a declarative probabilistic-logical formalism such as Markov logic, we are able to incorporate more semantic constraints and, therefore, extend traditional data quality rules. The results that we have presented in this paper indicate that taking a holistic view on data cleaning and that modeling this intuition within a Markov logic framework is a feasible and effective means to create data cleaning systems. \note{REWRITE THIS SENTENCE: With regards to modeling additional semantic constraints, larger and more heterogeneous datasets, present and future research will focus on the improving data quality for distributed data.}